\chapter{Case Study}
\label{chap:case_study}

In the following two sections, we present both a generalized and a specialized preprocessor for monitoring complex-typed event streams with MFODL and \MonPolyN. Both cases are related to \textsc{Dfinity}. They use MFOTL and \MonPoly to verify policies against JSON-formatted application logs collected during system tests \cite{IcPolicymonitoringMaster}. We then show how CMFODL can be used to simplify preprocessing and improve the readability and maintainability of the formulas.

\section{Generalized preprocessing}
Using this approach, we monitor any MFODL formulas against arbitrary JSON logs without adapting the preprocessing and the signatures to the particular use case. To achieve this, the generalized preprocessor transforms a stream of JSON objects to a temporal structure over the signature described in Listing \ref{fig:generalized_mfodl_signature}. For every encountered JSON document at time point $i$ in the input log, the preprocessor proceeds as follows: For every (possibly nested) JSON object, it generates a unique identifier $id$ and outputs the tuple $(id)$ on the relation $\mathsf{root}$. It then continues by traversing the fields of the object: for each field named $f$ of type $\mathsf{int}$, $\mathsf{float}$, $\mathsf{string}$, $\mathsf{bool}$ or $\mathsf{null}$, a new identifier $id_{val}$ is generated for the field value $v$ and written as the tuple $(id_{val}, v)$ (or $(id_{val})$ for constant values $\mathsf{true}$, $\mathsf{false}$ and $\mathsf{null}$) to the corresponding relations $\mathsf{int}$, $\mathsf{float}$, $\mathsf{string}$, $\mathsf{true}$, $\mathsf{false}$ or $\mathsf{null}$. Finally, the tuple $(id, f, id_{val})$ is written to the relation $\mathsf{key}$, introducing a reference from the value to its object under the field $f$. In the case of a field value being an object, the procedure is repeated recursively. JSON arrays are processed similarly to objects: Instead of writing a tuple to the $\mathsf{key}$ relation, we generate a tuple $(id, index, id_{val})$ in the relation $\mathsf{idx}$, establishing a reference between the array with identifier $id$ and the value with identifier $id_{val}$ of the element at index $index$. To generate the correct timestamp for each time point, the preprocessor accepts a JSON path to the (possibly nested) field containing the timestamp of each JSON log entry.

When providing the JSON log described in Listing \ref{fig:generalized_json_log} as input to the preprocessor, together with the timestamp path $\mathsf{event.log\_entry.time}$, the transformed log in Listing \ref{fig:generalized_output} is produced.

We formulate the following specification in MFODL over the representation of the temporal structure in Listing \ref{fig:generalized_output}: \textit{\enquote{Refute every program trace containing an event of type $\mathsf{Log}$ where $\mathsf{event.log\_entry.level}$ is equal to $\mathsf{ERROR}$ or $\mathsf{CRITICAL}$.}} Listing \ref{fig:generalized_formula} presents a possible MFODL formula for this specification.

We now solve the same task by formulating the specification in CMFODL instead and applying the extensions of \MonPoly presented in this thesis. We additionally make use of a minimal log preprocessor described in Appendix \ref{code:minimal_json_preprocessor} that extracts the timestamp of each log entry, such that the input log stream to \MonPoly conforms to the grammar described in Listing \ref{fig:json_log_ebnf}. The CMFODL signature is described in Listing \ref{fig:generalized_cmfodl_signature}, while the CMFODL formula is shown in Listing \ref{fig:generalized_cmfodl_formula}. First, the CMFODL signature describes the structure of the input log exactly, making it easier for users to understand the involved data structures. Comparing the MFODL formulas in Listing \ref{fig:generalized_formula} with the CMFODL formula in Figure \ref{fig:generalized_cmfodl_formula} shows that the same specification can be formulated much more naturally in CMFODL and therefore reduces the risk of introducing subtle bugs while formulating specifications.

\begin{figure}[p]
	\begin{lstlisting}[
		style=signatures,
		caption=Generalized MFODL signature,
		label=fig:generalized_mfodl_signature,
		literate=
			*{root(}{{{\color{codeblue}root}(}}{4}
			 {int(}{{{\color{codeblue}int}(}}{3}
			 {float(}{{{\color{codeblue}float}(}}{5}
			 {str(}{{{\color{codeblue}str}(}}{3}
			 {true(}{{{\color{codeblue}true}(}}{4}
			 {false(}{{{\color{codeblue}false}(}}{5}
			 {null(}{{{\color{codeblue}null}(}}{4}
			 {key(}{{{\color{codeblue}key}(}}{3}
			 {idx(}{{{\color{codeblue}idx}(}}{3}
	]
root(id:int)                                 (* object         *)
int(id:int, value:int)                       (* int value      *)
float(id:int, value:float)                   (* float value    *)
str(id:int, value:string)                    (* string value   *)
true(id:int)                                 (* constant true  *)
false(id:int)                                (* constant false *)
null(id:int)                                 (* constant null  *)
key(object:int, field:string, value:int)     (* object field   *)
idx(array:int,index:int,value:int)           (* array element  *)\end{lstlisting}
\end{figure}

\begin{figure}[p]
	\begin{lstlisting}[style=json,caption=Input JSON log,label=fig:generalized_json_log]
{
	'type':"Log",
	'event':{
		'src':{'a':0},
		'log_entry':{
			'level':"INFO",'time':1648053358,
			'message':"Configuration upated",'module':"auth",'line':17,
			'host':"127.0.0.1"
		}
	}
}
{
	'type':"Log",
	'event':{
		'src':{'a':0},
		'log_entry':{
			'level':"ERROR",'time':1648053380,
			'message':"Power loss",'module':"power",'line':567,
			'host':"127.0.0.1"
		}
	}
}\end{lstlisting}
\end{figure}

\begin{figure}[p]
	\begin{lstlisting}[
		style=signatures,
		caption=Output of the generalized preprocessor,
		label=fig:generalized_output,
		literate=
			*{root(}{{{\color{codeblue}root}(}}{4}
			 {int(}{{{\color{codeblue}int}(}}{3}
			 {float(}{{{\color{codeblue}float}(}}{5}
			 {str(}{{{\color{codeblue}str}(}}{3}
			 {true(}{{{\color{codeblue}true}(}}{4}
			 {false(}{{{\color{codeblue}false}(}}{5}
			 {null(}{{{\color{codeblue}null}(}}{4}
			 {key(}{{{\color{codeblue}key}(}}{3}
			 {idx(}{{{\color{codeblue}idx}(}}{3}
			 {@1648053358}{{{\color{codegreen}@1648053358}}}{11}
			 {@1648053380}{{{\color{codegreen}@1648053380}}}{11}
	]
@1648053358 root(0); key(0,"type",1)(3,"a",4)(2,"src",3)(5,"level",6)(5,"time",7)(5,"message",8)(5,"module",9)(5,"line",10)(5,"host",11)(2,"log_entry",5)(0,"event",2); int(4,0)(7,1648053358)(10,234); str(1,"Log")(6,"INFO")(8,"Configuration updated")(9,"auth")(11,"127.0.0.1")
@1648053380 root(12); key(12,"type",13)(15,"a",16)(14,"src",15)(17,"level",18)(17,"time",19)(17,"message",20)(17,"module",21)(17,"line",22)(17,"host",23)(14,"log_entry",17)(12,"event",14); int(16,0)(19,1648053380)(22,567); str(13,"Log")(18,"ERROR")(20,"Power loss")(21,"power")(23,"127.0.0.1")\end{lstlisting}
\end{figure}

\begin{figure}[p]
	\begin{lstlisting}[style=formula,caption=MFODL formula,label=fig:generalized_formula]
LET Log(level, message, host) = EXISTS r. root(r) AND
  (EXISTS t. key(r, "type", t) AND str(t, "Log")) AND
  (EXISTS e, le, lvl, msg, id. key(r, "event", e) AND
    key(e, "log_entry", le) AND
    key(le, "level", lvl) AND str(lvl, level) AND
    key(le, "message", msg) AND str(msg, message) AND
    key(le, "host", id) AND str(id, host))
IN
Log(level, message, host) IMPLIES
	(NOT level = "CRITICAL" AND NOT level = "ERROR")
	\end{lstlisting}\end{figure}

\begin{figure}[p]
	\begin{lstlisting}[
		style=signatures,
		caption=CMFODL signature,
		label=fig:generalized_cmfodl_signature,
		literate=
	     *{Log}{{{\color{codeblue}Log}}}{3}
	      {LogEvent}{{{\color{codeblue}LogEvent}}}{8}
	      {event}{{{\color{keyword}{event}}}}{5}
	      {event:}{{event:}}{6}
	      {:}{{{\color{codegray}{:}}}}{1}
	      {,}{{{\color{codegray}{,}}}}{1}
	      {\{}{{{\color{codegray}{\{}}}}{1}
	      {\}}{{{\color{codegray}{\}}}}}{1}
	      {[}{{{\color{codegray}{[}}}}{1}
	      {]}{{{\color{codegray}{]}}}}{1},
	]
LogEvent {
	src: {a: int},
	log_entry: {
		level: string,
		time: int,
		message: string,
		module: string,
		line: int,
		host: string
	}
}

event Log {
	type: string, event: LogEvent
}\end{lstlisting}
\end{figure}

\begin{figure}[p]
	\begin{lstlisting}[style=formula,caption=CMFODL formula,label=fig:generalized_cmfodl_formula]
LET is_error(event) =
	event.log_entry.level = "ERROR" OR
	event.log_entry.level = "CRITICAL"
IN
Log(l) AND l.type = "Log" IMPLIES NOT is_error(l.event)\end{lstlisting}
\end{figure}

\section{Specialized preprocessing}
We now look at the current approach of \textsc{Dfinity}, which utilizes a specialized preprocessor \cite{IcPolicymonitoringMaster}.  In this case, we do not intend to entirely replace the log preprocessing with CMFOTL since the preprocessing is needed to enrich the application logs with additional information required to verify the desired properties. Instead, we want to show that even in cases where log preprocessing is necessary, CMFODL still provides a benefit by allowing users to formulate specifications so that the resulting formulas naturally correspond to the underlying structure of the event data. This increases the readability and maintainability of formulas.

To clarify this argument, we look at the abstract application event $reboot$, referred to in multiple policies including $\mathsf{reboot\_count}$ \cite{IcPolicymonitoringMaster}. \textsc{Dfinity}'s current \textsc{Python}-based preprocessor generates a tuple $(host\_addr, data\_center\_prefix)$ of the relation $\mathsf{reboot}$ whenever the abstract event $reboot$ occurs, namely whenever a JSON log entry is encountered with the following properties: The field $\mathsf{\_source.host.ip}$ is equal to $host\_addr$, the field $\mathsf{\_source.syslog.identifier}$ is equal to the string "systemd", and the message of the log entry under $\mathsf{\_source.message}$ is equal to the string \enquote{Starting IC replica...}. In a corresponding MFODL formula, the occurrence of a reboot can therefore be verified by the predicate formula $\mathsf{reboot(host\_addr, data\_center\_prefix)}$.

With CMFODL, we achieve the same result by introducing a custom predicate as shown in Listing \ref{fig:reboot_abstraction}. The advantage of the latter approach is that we can formalize the same abstractions of events by using a single language in a single place without the need for synchronization. This improves the readability and comprehensibility of the formulas. While the CMFODL formula in Listing \ref{fig:reboot_abstraction} may look more complex than the original MFODL formula, it is comparable to the corresponding \textsc{Python}-code of the preprocessor \cite{IcPolicymonitoringMaster}. In addition, the concept of CMFODL modules could be introduced to improve the reusability of policies.

\begin{figure}
	\begin{lstlisting}[style=formula,caption=CMFODL formula abstracting a reboot event, label=fig:reboot_abstraction]
LET reboot(event, host_addr, data_center_prefix) =
	event._source.host.ip = host_addr AND
	event._source.host.data_center_prefix = data_center_prefix AND
	event._source.syslog.identifier = "systemd" AND
	event._source.message = "Starting IC replica..."
IN
Event(e) AND reboot(e, addr, prefix)\end{lstlisting}
\end{figure}

%%% Local Variables:
%%% mode: latex
%%% TeX-master: "thesis"
%%% End:
