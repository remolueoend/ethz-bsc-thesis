\chapter{Complex-Typed MFODL}
\label{chap:complex_data_types}

As outlined in the introduction, \MonPoly makes use of MOFDL as its policy specification language \cite{basinMonPolyMonitoringTool2017}. To support monitoring complex-structured application logs using MFODL policies, we propose the extension CMFODL of MFODL with support for complex data types. Section \ref{sec:extensions_mfold_syntax} describes the abstract syntax of CMFODL, whereas Section \ref{sec:mfodl_semantics} presents the semantics of the newly introduced syntax.

\section{Abstract syntax}
\label{sec:extensions_mfold_syntax}
We first introduce the syntax of complex-typed metric first-order dynamic logic (CMFODL). Here we diverge from the commonly used, minimalistic single-sorted definition~\cite{basinMonitoringMetricFirstorder2015} and opt to cover additional details as specified in the monitoring tool \MonPoly~\cite{basinMonPolyMonitoringTool2017}.

A~sorted first-order signature defines constant, function,
and predicate symbols by enumerating their names and respective sorts.
Intuitively, a sort describes a subset of domain values
that will be used to interpret the symbols in the semantics.
The signature assigns a single sort to each constant symbol.
Predicate and function symbols require a sequence of sorts that
characterize their arguments; function symbols have an additional sort for their result.
Here we opt not to include function symbols in the signature. We incorporate
them into the syntax and define their sorts indirectly using a type system later.

We fix an infinite set of names $\mfodlNames$ and a set $\mfodlPrimitiveSorts =
	\{\mfodlInt,\mfodlFloat,\mfodlString,\mfodlRegex,\mfodlBool, \mfodlNull\}$ of
six primitive sorts: integers ($\mfodlInt$), floats ($\mfodlFloat$), strings ($\mfodlString$), regular expressions ($\mfodlRegex$) booleans ($\mfodlBool$) and the singleton sort $\mfodlNull$ consisting of the single constant value $\mathsf{null}$. We assume that $\mfodlNames$ and $\mfodlPrimitiveSorts$ are disjoint.
%The set $\mfodlPrimitiveSorts\subseteq\Sorts$ denotes all primitive sorts $\{\mfodlInt,\mfodlFloat,\mfodlString,\mfodlRegex\}$.
A~sorted first-order signature $\mfodlSig$ is a tuple \((\mfodlSigSorts, \mfodlSigSdefs, \mfodlSigConsts, \mfodlSigPreds)\), %\subseteq 2^{\Sorts}\times
%2^{\mfodlNames\times\Sorts} \times 2^{\mfodlNames\times \mfodlPrimitiveSorts^*}\).
where $\mfodlSigSorts$ is a finite subset of $\mfodlNames \cup \mfodlPrimitiveSorts$, presenting the sorts used by the signature.
The partial mapping $\mfodlSigSdefs$ defines the non-primitive sorts; we will explain it below.
The partial mapping $\mfodlSigConsts \in \mfodlNames \mfodlpto \mfodlSigSorts$ assigns a sort to each constant symbol.
The available constant symbols are implicitly given by $\dom(\mfodlSigConsts)$, the domain of $\mfodlSigConsts$.
Finally, $\mfodlSigPreds \in \mfodlNames \mfodlpto \mfodlPrimitiveSorts^*$ assigns a list of (primitive) sorts
to each predicate symbol in $\dom(\mfodlSigPreds)$.
%
For example, $\mfodlSig =
	(\mfodlPrimitiveSorts, \emptyset, \{\mathsf{r} \mapsto \mfodlInt\}, \{\mathsf{publish} \mapsto (\mfodlInt),\mathsf{approve} \mapsto (\mfodlInt)\})$ defines the
signature for a simple reporting system, where relevant predicate symbols correspond to publishing
and approving reports with integer IDs, and we have an integer constant $\mathsf{r}$.
There is no custom sort; hence we specified the empty mapping $\emptyset$ for $\mfodlSigSdefs$.

We allow the definition of custom-named sorts in the signature.
%
In particular, named products of sorts with named fields can be defined as a custom sort.
The set of names $\mfodlNames$ can be used for custom sort names and their field
names.
The structure of a custom sort is described by a \emph{sort schema}.
The set $\mfodlSchemas$ of all sort schemas is the smallest set such that if $f$ is a function from a finite (possibly empty) subset of $\mfodlNames$ to $\mfodlNames \cup \mfodlPrimitiveSorts$, then $\Pi f \in \mfodlSorts$. Note that $\Pi$ is just a symbol that marks the schema as a product. We usually write product schemas by enumerating the pairs of field names and their sorts, e.g. $\{n_1: s_1, n_2: s_2, \dots, n_k: s_k\}$ is a product schema with $k$ fields. If $s$ is a sort schema, we write $\ran(s)$ for the set of sort names occurring in $s$. For example, $\ran(\Pi f) = \ran(f)$, the range of $f$. The sort definition $\mfodlSigSdefs$ is now a partial mapping $\mfodlCustomSorts \mfodlpto \mfodlSchemas$ with $\mfodlCustomSorts = \mfodlSigSorts - \mfodlPrimitiveSorts$, subject to a well-formedness constraint. It ensures that all sorts used in sort definitions are defined and that there is no recursion between sorts.
More precisely, there must be a sequence $S_1 \subset S_2 \subset \cdots \subset S_k$ such that
\begin{compactitem}
	\item $S_1 \subseteq \mfodlPrimitiveSorts$,
	\item $S_k = \dom(\mfodlSigSdefs)$, and
	\item $S_{i+1}\setminus S_i = \{s_i\}$ for all $i < k$ and some $s_i$, where $\ran(\mfodlSigSdefs(s_i)) \subseteq S_i$.
\end{compactitem}

Since we allow custom products of sorts, one can easily encode predicate symbols using this more
general mechanism. We therefore simplify the signature to the triple $(\mfodlSigSorts, \mfodlSigSdefs, \mfodlSigConsts)$.
For the reporting system example above, the new signature is
$(\mfodlPrimitiveSorts\cup\{\p{publish},\p{approve}\}, \{\p{publish} \mapsto \{\p{\_1}: \mfodlInt\}, \p{approve} \mapsto \{\p{\_1}: \mfodlInt\}\},
	\{\p{r} \mapsto \mfodlInt\})$.
It has two custom sorts, one for each predicate symbol, that are defined as products with a single field $\p{\_1}$.
In the following, we assume that $\mfodlNames$ contains a name of the form $\p{\_}k$ for every natural number $k \geq 1$.
Thus we can perform the encoding for predicates of any arity.
A~sort $s$ such that $\mfodlSigSdefs(s)$ is a product (exclusively) over fields $\p{\_1}$ to $\p{\_}k$ is called a \emph{tuple sort}.
The particular case of an empty product is also a tuple sort.
We define the arity $\iota(s)$ of a tuple sort $s$ to be $n$ and leave it undefined for all other sorts.

To define the syntax of formulas, we further fix a countably infinite set \(\mfodlVars\) of variables and the set \(\mfodlIntervals\) of nonempty intervals \([a,b) := \{x \in \mfodlNat \mid a \leq x < b\}\), where \(a \in \mfodlNat\), \(b \in \mfodlNat \cup \{\infty\}\), and \(a < b\). Given a signature $(\mfodlSigSorts,\mfodlSigSdefs,\mfodlSigConsts)$, formulas \(\varphi\) are defined inductively, where $s$, $c$, \(v\), \(n\), \(\bar{v}\), \(\bar{nv}\), and \(I\) range over $\mfodlSigSorts$, $\dom(\mfodlSigConsts)$, $\mfodlVars$, \(\mfodlNames\), \(\mfodlVars^*\),\((\mfodlNames\times\mfodlVars)^*\), and \(\mfodlIntervals\), respectively. Figure \ref{fig:mfodl_syntax_def} formalizes syntax of MFODL, where $\Omega\in\{\mfodlaggcnt,\mfodlaggsum,\mfodlaggavg,\mfodlaggmed,\mfodlaggmin,\mfodlaggmax\}$ stands for an aggregation operator. $t$ relates to terms, $\phi$ to formulas, and $r$ to regular expressions over program traces. Sort $s$ in a subformula of the form $\mfodlLetin{s\{n:t,\ldots, n:t\}}{\phi}{\psi}$ or $\mfodlLetin{n(\bar{v})}{\phi}{\phi}$ cannot occur in $\phi$ or outside of the subformula. Let $\mfodlfv{\phi}$ denote a set of free variables of an MFODL formula $\phi$, defined as usual, whereas the set $\mfodlTerms$ represents the set of all MFODL terms.


\begin{figure}
	\[
		\begin{array}{r c l l}
			% c& :: = & i \mid f \mid s &\qquad\qquad\qquad\quad\!//\text{ Constants}\\
			t    & :: = & c \mid v \mid \mfodlfti{t} \mid \mfodlitf{t} \mid \mfodlrts{t} \mid \mfodlstr{t} \mid \mathsf{i2s}(t) \mid \mathsf{s2i}(t) \mid \mathsf{f2s}(t) \mid \mathsf{s2f}(t)                                                                          \\
			     &      & -t \mid t + t \mid t - t \mid t * t \mid t / t \mid t \% t \mid
			t.n \mid s\overbrace{\{n:t,\ldots, n:t\}}^{= \mfodlSigSdefs(s)}                                                                                                                                                                                                                   \\
			\phi & ::=  & s(t)  \mid \mfodlclsymb{s}(t) \mid                                                                                                                                                                                                            \\
			     &      & \bot \mid \top \mid s(\overbrace{t,\ldots,t}^{\iota(s)}) \mid t \approx t \mid t \prec t \mid t \preceq t \mid t\ \mfodlsubs\ t \mid t\ \mfodlmatches\ t(t,\ldots,t) \mid                                                                     &                     \\
			     &      & \neg \phi \mid \phi \land \phi \mid  \phi \lor \phi \mid \phi \rightarrow \phi \mid \phi \leftrightarrow \phi \mid \exists\bar{v}.\ \phi \mid  \forall\bar{v}.\ \phi \mid                                                                     &                     \\
			     &      & \mfodlPrev_I \phi \mid \mfodlNext_I \phi \mid \mfodlOnce_I\phi \mid \mfodlEventually_I\phi \mid \mfodlPAlways_I\phi\mid \mfodlAlways_I\phi\mid\phi\mfodlSince_I \phi \mid \phi \mfodlUntil_I \phi \mid \matchMP{r}{I} \mid \matchMF{r}{I}\mid &                     \\
			     &      & v \leftarrow \Omega \ v; \bar{v} \ \phi\mid \mfodlLetin{s\overbrace{\{n:v,\ldots, n:v\}}^{= \mfodlSigSdefs(s)}}{\phi}{\phi}\mid \mfodlLetin{n(\bar{v})}{\phi}{\phi}                                                                           &                     \\
			r    & ::=  & \cdot \mid \phi? \mid r \ r \mid r + r \mid r^*                                                                                                                                                                                               
		\end{array}
	\]
	\caption{MFODL syntax definition}
	\label{fig:mfodl_syntax_def}
\end{figure}


\section{Semantics}
\label{sec:mfodl_semantics}
MFODL formulas are interpreted over \emph{temporal structures (TS)}, which model timestamped and totally ordered sequences of observations.

Given a signature $\Delta$, each observation in TS consists of a timestamp and a finite first-order $\Delta$-structure $\mathcal{D}$, which interprets (i.e., associates appropriate values to) the elements of the signature. In $\mathcal{D}$ each sort $s\in S$ is interpreted with a nonempty domain $d$ of values each tagged with its sort, i.e., $s^{\mathcal{D}} = d \mfodlImage \lambda e.\; (s, e)$. In particular, primitive sorts have the expected domains: the sort of integers is interpreted as the set of tagged integers ($\mfodlInt^{\mathcal{D}}=\mfodlIntDom$). Similar holds for floats ($\mfodlFloat^{\mathcal{D}}=\mfodlFloatDom$), strings ($\mfodlString^{\mathcal{D}}=\mfodlStringDom$), and regular expressions ($\mfodlRegex^{\mathcal{D}}=\mfodlRegexDom$). The interpretation function $\cdot^{\mathcal{D}}$ is lifted for set operations like unions and products. To define the interpretation of a custom product sort $s \in \mfodlSigSorts-\mfodlPrimitiveSorts$, we define $\mfodlSigSdefs(s)=\Pi f$ as $\left[\prod_{n\in \dom(f)} f(n)^{\mathcal{D}} \right] \mfodlImage \lambda e.\; (s, e)$.

Given a signature $\Delta = (S, S_{def}, C)$, the finite first-order $\Delta-$structure is the triple
$\mathcal{D}=(\mfodlDomain, \mathbb{C}, \mathbb{O})$ where $\mfodlDomain=\bigcup_{s\in S} s^\mathcal{D}$ is
union of values of all domains. For every $(c,s)\in C$ we have $c^\mathcal{D}\in s^\mathcal{D}$ in
$\mathbb{C}$ and we have a finite set of objects $\mathbb{O}\subseteq \mfodlDomain$. We denote a closure of
$\mathbb{O}$ with respect to the projection operator ($.$) as the smallest set
$\mathbb{O}\downarrow$ such that:
\begin{compactitem}
	\item $\mathbb{O}\subseteq \mathbb{O}\downarrow$
	\item if $(e,f)\in \mathbb{O}\downarrow$ then $o\in \mathbb{O}\downarrow$ for all $(\_,o)\in f$
\end{compactitem}

A temporal structure \(\rho\) is then an infinite sequence (or a stream) \((\tau_i, \mathcal{D}_i)_{i \in \mfodlNat}\) of finite first-order $\Delta$-structures $\mathcal{D}_i$ with associated time-stamps $\tau_i$. We refer to a component of a structure at a specific time-point $i$ using a subscript (e.g., $\mathbb{D}_i$). All finite first-order $\Delta$-structures $\mathcal{D}_i$ agree on their sorts ($\forall i.\ \forall s\in S.\ s^{\mathcal{D}_{i}}=s^{\mathcal{D}_{i+1}}$) and constant interpretations ($\forall i.\ \mathbb{C}_{i}=\mathbb{C}_{i+1}$). Since they agree on their sorts, they also agree on their domains ($\forall i.\ \mfodlDomain_i=\mfodlDomain_{i+1}$), When referring to these common components in a TS, we omit the reference to a particular time-point (e.g., we write just $\mfodlDomain$).

Time-stamps are discrete, modeled as natural numbers \(\tau_i \in \mfodlNat\). We allow the event source to use finitely precise clocks:
structures at different time-points $i \neq j$ may have the same time-stamp $\tau_i = \tau_j$. The sequence of time-stamps must be non-strictly increasing ($\forall i.\;\tau_i \leq \tau_{i+1}$) and always eventually strictly increasing ($\forall \tau.\;\exists i.\ \tau < \tau_i$).

The valuation \(v\) is a mapping \(\mfodlVars \to \mfodlDomain\), assigning domain elements to variables. Overloading notation, $v$ can be applied to terms:
\begin{compactitem}
	\item $v(c)\in \mfodlDomain$;
	\item $v(\mfodlfti{t})=\mfodlffti{v(t)}$; $v(\mfodlitf{t})=\mfodlfitf{v(t)}$;
	\item $v(\mfodlrts{t})=\mfodlfrts{v(t)}$; $v(\mfodlstr{t})=\mfodlfstr{v(t)}$;
	\item $v(\mathsf{i2s}(t))=\mathsf{int\_to\_str}(v(t))$; $v(\mathsf{s2i}(t))=\mathsf{str\_to\_int}(v(t))$;
	\item $v(\mathsf{f2s}(t))=\mathsf{float\_to\_str}(v(t))$; $v(\mathsf{s2f}(t))=\mathsf{str\_to\_float}(v(t))$;
	\item $v(-t)=-v(t)$;
	\item $v(t\bowtie t')=v(t)\bowtie v(t')$ for $\bowtie\in \{+,-,*,/,\%\}$;
	\item $v(t.f) = d$ if $v(t) = (n,o)$ and $(f,d)\in o$.
	\item $v(s\{n_1:t_1,\ldots,n_k:t_k\}) = (s,\{(n_i,v(t_i)) \mid i = 1,\ldots, k\})$
\end{compactitem}

We write \(v[x \mapsto y]\) for the function equal to \(v\), except that the argument \(x\) is mapped to \(y\). For a vector of free variables $\bar{u}=[u_1,\ldots,u_k]$, we write \(v(\bar{u})\) for the tuple \((v(u_1),\ldots v(u_k))\). We also define a let mapping $\delta$ as a partial function $\delta:\mfodlNames \rightharpoonup (\mfodlNat \rightarrow 2^{\mfodlDomain})$.

Figure \ref{fig:mfodl_operator_semantics} describes the semantics of the operators defined in the MFODL syntax. The relation \(\delta, v,\,i \models_\rho \varphi\) defines the satisfaction of the formula \(\varphi\) for a let mapping $\delta$, valuation $v$, and a time-point \(i\) with respect to the temporal structure $\rho$. Whenever~$\rho$ is fixed and clear from the context, we omit the subscript on $\models$. We focus only on the core subset of the operators, while the rest of the operators are shorthands. Temporal operators with no interval have $[0,\infty)$ instead.

\begin{figure}
	\[
		\begin{array}{@{}ll@{\;\;}@{\;\;}ll@{}}
			\delta,\,v,\,i \models s(t)                                   & \text{iff } v(t)=(s,\_) \text{ and if } \delta(\p{s})(i) \neq \bot \text{ then }                                                   \\ & v(t)\in\delta(\p{s})(i) \text{ else } v(t)\in \mathbb{O}^{\mathcal{D}_i}                     &  & \\
			\delta,\,v,\,i \models \mfodlcl{s}(t)\qquad\qquad\quad        & \text{iff } v(t)=(s,\_) \text{ and if } \delta(\p{s})(i) \neq \bot \text{ then }                                                   \\ & v(t)\in\delta(\p{s})(i)\downarrow \text{ else } v(t)\in \mathbb{O}^{\mathcal{D}_i}\downarrow &  & \\

			\delta,\,v,\,i \models t \approx t'                           & \text{iff } v(t) = v(t')                                                                                                     &   & \\
			\delta,\,v,\,i \models t \preceq t'                           & \text{iff } v(t) \leq v(t')                                                                                                  &   & \\
			\delta,\,v,\,i \models t \mfodlmatches\; t' (t_1,\ldots, t_n) & \text{iff } v(t) \text{ matches } v(t')
			\text{with capture group } i \text{ valued as } v(t_i)        &                                                                                                                              &     \\
			\delta,\,v,\,i \models \lnot \varphi                          & \text{iff } \delta,\,v,\,i \not\models \varphi                                                                               &   & \\
			\delta,\,v,\,i \models \varphi \land \psi                     & \text{iff } \delta,\,v,\,i \models \varphi \text{ and } \delta,\,v,\,i \models \psi                                          &   & \\
			\delta,\,v,\,i \models \exists x.\;\varphi                    & \text{iff } \delta,\,v[x \mapsto z],\,i \models \varphi \text{ for some \(z \in \mfodlDomain\)}                              &   & \\
			\delta,\,v,\,i \models \mfodlPrev_I \varphi                   & \text{iff \(i > 0\), \(\tau_i - \tau_{i-1} \in I\), and } \delta,\,v,\,i-1 \models \varphi                                   &   & \\
			\delta,\,v,\,i \models \mfodlNext_I \varphi                   & \text{iff \(\tau_{i+1} - \tau_i \in I\) and } \delta,\,v,\,i+1 \models \varphi                                               &   & \\
			\delta,\,v,\,i \models \varphi \mfodlSince_I \psi             & \multicolumn{3}{l@{}}{\text{iff } \delta,\,v,\,j \models \psi \text{ for some \(j \leq i\), \(\tau_i - \tau_j \in I\),}}           \\
			                                                              & \text{and } \delta,\,v,\,k \models \varphi \text{ for all \(k\) with \(j < k \leq i\)}                                       &   & \\
			\delta,\,v,\,i \models \varphi \mfodlUntil_I \psi             & \multicolumn{3}{l@{}}{\text{iff } \delta,\,v,\,j \models \psi \text{ for some \(j \geq i\), \(\tau_j - \tau_i \in I\), }}          \\
			                                                              & \text{and } \delta,\,v,\,k \models \varphi \text{ for all \(k\) with \(i \leq k < j\)}                                       &   & \\
			\delta,\,v,\,i \models \mfodlLetin{s\{\bar{nu}\}}{\phi}{\psi} & \multicolumn{3}{l@{}}{\text{iff } \delta[s \rightarrow f],\,v,\,i \models \psi\text{ where} }                                      \\
			                                                              & f=\lambda i.\{(s,\{n_1:w(u_1),\ldots,n_k:w(u_k)\}) \mid \delta,\,w,\,i \models \phi\}                                        &   & \\
			\delta,\,v,\,i \models \matchMP{r}{I}                         & \multicolumn{3}{l@{}}{\text{iff } \tau_j - \tau_i \in I \text{ and } (i, j)\in \mathcal{R}(r) \text{ for some } j \geq i   }       \\
			\delta,\,v,\,i \models \matchMF{r}{I}                         & \multicolumn{3}{l@{}}{\text{iff } \tau_i - \tau_j \in I \text{ and } (j, i)\in \mathcal{R}(r) \text{ for some } j \leq i   }       \\
		\end{array}
	\]
	\[
		\begin{array}{@{}ll@{\;\;}@{\;\;}ll@{}}
			\mathcal{R}(\star)     & = \{ (i, i+1) \mid i \in \mathds{N} \}                                                                                                                                         \\
			\mathcal{R}(\phi?)     & = \{ (i, i) \mid \delta,\ v,\ i \models \phi \}                                                                                                                                \\
			\mathcal{R}(r + s)     & = \mathcal{R}(r) \cup \mathcal{R}(s)                                                                                                                                           \\
			\mathcal{R}(r \cdot s) & = \left\{ (i, k) \mid \exists j.\ (i,j) \in \mathcal{R}(r) \text{ and } (j,k) \in \mathcal{R}(s) \right\}                                                                      \\
			\mathcal{R}(r*)        & = \left\{ (i, i) \mid i \in \mathds{N} \right\} \cup \left\{ (i_0,i_k) \mid \exists i_1,\dots, i_k .\ (i_j, i_{j+1}) \in \mathcal{R}(r) \text{ for all } 0 \leq j < k \right\}
		\end{array}
	\]
	$x \prec y := x \preceq y \land \neg x \approx y$,
	$\bot := 0 \approx 1$,
	$\top := \neg \bot$,
	$\p{s}(t_1,\ldots,t_n) = \p{s}(\p{s}\{\p{\_1}:t_1,\ldots,\p{\_n}:t_n\})$,
	$t\ \mfodlsubs\ t' := t\ \mfodlmatches\; t'()$,
	$\varphi \lor \psi := \neg \varphi \land \neg \psi$,
	$\varphi \rightarrow \psi := \neg \varphi \vee \psi$,
	$\varphi \leftrightarrow \psi := \varphi \rightarrow \psi \land \psi \rightarrow \varphi$,
	$\forall x.\, \varphi := \neg \exists x.\, \neg \varphi$,
	$\mfodlOnce_I \varphi := \top \mfodlSince_I \varphi$,
	$\mfodlEventually_I \varphi := \top \mfodlUntil_I \varphi$,
	$\mfodlAlways_I \varphi := \neg \mfodlEventually_I \neg \varphi$,
	$\mfodlPAlways_I \varphi := \neg \mfodlOnce_I \neg \varphi$, and
	$\mfodlLetin{\p{p}(t_1,\ldots,t_n)}{\phi}{\psi} := \mfodlLetin{\p{p}(\{\p{\_1}:t_1,\ldots,\p{\_n}:t_n\})}{\phi}{\psi}$.
	\caption{MFODL operator semantics}
	\label{fig:mfodl_operator_semantics}
\end{figure}





%%% Local Variables:
%%% mode: latex
%%% TeX-master: "thesis"
%%% End:
